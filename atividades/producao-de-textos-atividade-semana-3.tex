\documentclass[
	12pt,				% tamanho da fonte
	openright,			% capítulos começam em pág ímpar (insere página vazia caso preciso)
	oneside,			% para impressão em recto e verso. Oposto a oneside
	a4paper,			% tamanho do papel.
	english,			% idioma adicional para hifenização
	french,				% idioma adicional para hifenização
	spanish,			% idioma adicional para hifenização
	brazil				% o último idioma é o principal do documento
	]{abntex2}

% ---
% Pacotes básicos
% ---
\usepackage{lmodern}			% Usa a fonte Latin Modern
\usepackage[T1]{fontenc}		% Selecao de codigos de fonte.
\usepackage[utf8]{inputenc}		% Codificacao do documento (conversão automática dos acentos)
\usepackage{lastpage}			% Usado pela Ficha catalográfica
\usepackage{indentfirst}		% Indenta o primeiro parágrafo de cada seção.
\usepackage{color}				% Controle das cores
\usepackage{graphicx}			% Inclusão de gráficos
\usepackage{microtype} 			% para melhorias de justificação
% ---

% ---
% Pacotes de citações
% ---
\usepackage[brazilian,hyperpageref]{backref}	 % Paginas com as citações na bibl
\usepackage[alf]{abntex2cite}	% Citações padrão ABNT

\usepackage{lipsum}				% para geração de dummy text

% ---
% Informações de dados para CAPA e FOLHA DE ROSTO
% ---
\titulo{Atividade avaliativa da semana 3}
\autor{Carlos Antonio Marques Maniero --- 1824312}
\local{UNIVESP}
\data{2018}
\orientador{Profa. Dra. Silvia M. Gasparian Colello}
\instituicao{%
  Univesidade Virtual do Estado de São Paulo --- UNIVESP
  \par
  Curso de Engenharia da Computação
  \par
  Diciplina de Produção de Textos
}
\tipotrabalho{Atividade Avaliativa}
% O preambulo deve conter o tipo do trabalho, o objetivo,
% o nome da instituição e a área de concentração
\preambulo{Atividade avaliativa apresentado como
    exigência parcial para Avaliação do
    curso da Diciplina de Produção de textos
    do curso de Engenharia da Computação pela UNIVESP.}
% ---


% ---
% Configurações de aparência do PDF final

% alterando o aspecto da cor azul
\definecolor{blue}{RGB}{41,5,195}

% informações do PDF
\makeatletter

\hypersetup{pdftitle={\@title},
            pdfauthor={\@author},
            pdfsubject={\imprimirpreambulo},
            pdfcreator={LaTeX with abnTeX2},
            pdfkeywords={abnt}{latex}{abntex}{abntex2}{trabalho acadêmico},
            colorlinks=true,       		% false: boxed links; true: colored links
            linkcolor=blue,          	% color of internal links
            citecolor=blue,        		% color of links to bibliography
            filecolor=magenta,      		% color of file links
            urlcolor=blue,
            bookmarksdepth=4}
\makeatother
% ---

% ---
% Espaçamentos entre linhas e parágrafos
% ---

% O tamanho do parágrafo é dado por:
\setlength{\parindent}{1.3cm}

% Controle do espaçamento entre um parágrafo e outro:
\setlength{\parskip}{0.2cm}  % tente também \onelineskip

% ---
% compila o indice
% ---
\makeindex
% ---

% ----
% Início do documento
% ----
\begin{document}

% Seleciona o idioma do documento (conforme pacotes do babel)
%\selectlanguage{english}
\selectlanguage{brazil}

% Retira espaço extra obsoleto entre as frases.
\frenchspacing

% ----------------------------------------------------------
% ELEMENTOS PRÉ-TEXTUAIS
% ----------------------------------------------------------
% \pretextual

% ---
% Capa
% ---
\imprimircapa{}
% ---

% ---
% Folha de rosto
% (o * indica que haverá a ficha bibliográfica)
% ---
\imprimirfolhaderosto*
% ---

% ---
% Epígrafe
% ---
\begin{epigrafe}
    \vspace*{\fill}
	\begin{flushright}
        \begin{adjustwidth}{7.5cm}{}
        \textit{``A língua materna, seu vocabulário e sua estrutura gramatical,
        não conhecemos por meio de dicionários ou manuais de gramática,
        mas graças aos enunciados concretos que ouvimos e reproduzimos
        na comunicação efetiva com as pessoas que nos rodeiam.}
        \end{adjustwidth}
        \textit{Mikhail Bakhtin}
	\end{flushright}
\end{epigrafe}
% ---

% ---
% inserir o sumario
% ---
\pdfbookmark[0]{\contentsname}{toc}
\tableofcontents*
\cleardoublepage{}
% ---

% ----------------------------------------------------------
% ELEMENTOS TEXTUAIS
% ----------------------------------------------------------
\textual{}

% ---
% Exercício 1
% ---
\chapter{Exercício 1}

Nas primeiras três semanas da disciplina, estudamos conceitos
relacionados ao tema da língua.
Com base no que foi visto, explique os seguintes conceitos:

\section{Letramento}
Letramento é definido por \hbox{\cite{soares}} como
``Estado ou condição de quem não apenas sabe ler e escrever,
mas cultiva e exerce as práticas sociais que usam a escrita'',
ou seja, o estado ou condição além da decodificação em que o
indivíduo torna-se usuário da lingua escrita usando-a como
ferramenta para suas praticas sociais.

\section{Analfabetismo funcional}
Podemos compreender o analfabetismo funcional como o não letramento.

\par

O indivíduo que, sendo capaz de decodificar a língua escrita
não conhece suas estruturas, nem é capaz de exercer suas praticas
sociais.

\section{Gramática descritiva}
``É a que orienta o trabalho dos linguistas, cuja preocupação é
descrever e/ou explicar as línguas tais como elas são faladas.''
\hbox{\cite{possenti}}

\par

A gramática descritiva, descreve a língua falada, mesmo que para
tal seja necessário ferir a norma culta, não estando portanto
em conformidade com a gramática normativa.

\section{Língua portuguesa de expressão brasileira}
A língua é um elemento vivo da nossa cultura. O Brasil é uma
grande intersecção de culturas com elementos oriundos da
cultura africana devido ao modelo escravocrata aqui adotado no passado,
de diversas regiões europeias por meio do processo imigratório e indígena
que aqui habitavam antes da chegada dos portugueses.

\par

Essa intersecção gerou o que chamamos de ``expressão brasileira'',
que embora siga o mesmo conjunto de normas da língua portuguesa de
expressão portuguesa, possui sua própria identidade.
\par

Podemos notar essas variações linguísticas mesmo dentro de um mesmo
estado. O Português falado na região central de São Paulo, por exemplo,
tem expressão diferente em comparação ao interior do estado.

\section{Concepção dialógica de língua}

A concepção dialógica de língua tem como preceito que todo discurso
se apoia no discurso de outrem.

\begin{citacao}
A orientação dialógica é naturalmente um fenômeno próprio a
todo discurso. Trata-se da orientação natural de qualquer discurso
vivo. Em todos os seus caminhos até o objeto, em todas
as direções, o discurso se encontra com o discurso de outrem
e não pode deixar de participar, com ele, de uma interação
viva e tensa. Apenas o Adão mítico que chegou com a primeira
palavra num mundo virgem, ainda não desacreditado, somente
este Adão podia realmente evitar por completo esta mútua
orientação dialógica do discurso alheio para o objeto. Para o
discurso humano, concreto e histórico, isso não é possível:
só em certa medida e convencionalmente é que pode dela se
afastar.
\hbox{\cite{bakhtin}}
\end{citacao}

% ---
% Exercício 2
% ---
\chapter{Exercício 2}

Escreva uma frase que exemplifique as seguintes funções da língua:

\section{Função conativa}

A função conativa tem propósito apelativo, por isso, pode também
ser definida como função apelativa. Tem como principal característica
o intuito de convencer o interlocutor.

\par
Alguns exemplos são: \textit{``Pé de pai pede Rider''} e
\textit{``Fisk, todo mundo fala bem''}.

\section{Função emotiva}
A função emotiva como o próprio nome sugere, usa da subjetividade com
a finalidade de emocionar o interlocutor.

% ---
% Exercício 3
% ---
\chapter{Exercício 3}

A linguagem pode se manifestar de diversas formas.
Explique a diferença entre a linguagem sonora e a linguagem verbal.

\section{Linguagem sonora e Linguagem Verbal}
A linguagem verbal utiliza de palavras na comunicação.
\par
A linguagem sonora está cada vez mais presente no nosso dia-a-dia.
Ao receber uma mensagem no celular, o usuário do dispositivo é
notificado através de um sinal sonoro e isso já o o suficiente
para o interlocutor entender a mensagem, não há necessidade de
palavras nesse contexto, um simples sinal sonoro já é o suficiente.

% ----------------------------------------------------------
% ELEMENTOS PÓS-TEXTUAIS
% ----------------------------------------------------------
\postextual{}

% ----------------------------------------------------------
% Referências bibliográficas
% ----------------------------------------------------------
\bibliography{producao-de-textos-atividade-semana-3}

\end{document}
